\documentclass[11pt,a4paper]{article}

\usepackage[utf8]{inputenc}
\usepackage[T1]{fontenc}
\usepackage{amsmath,amssymb}
\usepackage{booktabs}
\usepackage{array}
\usepackage{hyperref}
\usepackage{xcolor}
\usepackage{geometry}
\usepackage{enumitem}
\usepackage{fancyhdr}
\usepackage{listings}
\usepackage{framed}

\geometry{margin=1in}

\definecolor{darkblue}{RGB}{0,51,102}
\definecolor{linkblue}{RGB}{0,102,204}
\definecolor{codebg}{RGB}{245,245,245}
\definecolor{shadecolor}{RGB}{240,240,255}

\hypersetup{
    colorlinks=true,
    linkcolor=darkblue,
    urlcolor=linkblue
}

\lstset{
    backgroundcolor=\color{codebg},
    frame=single,
    basicstyle=\ttfamily\small,
    breaklines=true
}

\pagestyle{fancy}
\fancyhf{}
\rhead{YYSFold System Walkthrough}
\rfoot{Page \thepage}

\title{\textbf{YYSFold System Walkthrough}\\[0.5em]
\large A Complete Guide to the Dashboard and Pipeline}
\author{YYSFold Team}
\date{\today}

\begin{document}

\maketitle

\begin{abstract}
This document provides a comprehensive walkthrough of the YYSFold blockchain fingerprinting system. It explains what you see on the live dashboard, how each component works, and the complete data pipeline from raw blockchain data to behavioral fingerprints.
\end{abstract}

\tableofcontents
\newpage

\section{What Is YYSFold?}

YYSFold is a \textbf{blockchain behavioral fingerprinting system}. It watches blockchain activity in real-time, compresses each block into a compact ``fingerprint,'' identifies unusual patterns (hotzones), and can predict what the next block might look like---all while generating cryptographic proofs that the analysis was done correctly.

\begin{shaded}
\textbf{Live Demo:} \url{https://yysfold.ngrok.io}
\end{shaded}

\section{Dashboard Overview}

\subsection{Header and Navigation}

The header contains:
\begin{itemize}
    \item \textbf{YYSFold logo} --- Links to home
    \item \textbf{Atlas} --- 2D/3D visualization of the vector space
    \item \textbf{Chat} --- AI assistant for querying block data
\end{itemize}

\subsection{Latest Block Fingerprint Card}

The main card shows the most recent block fingerprint with:

\begin{center}
\begin{tabular}{|l|p{8cm}|}
\hline
\textbf{Element} & \textbf{Description} \\
\hline
sol $\cdot$ \#383106448 & Chain (Solana) and block height \\
\hline
{[Live]} & Real-time SSE connection active \\
\hline
Updated 7s & Time since last heartbeat \\
\hline
Tags (pills) & Semantic behaviors detected in this block \\
\hline
Block hash & Truncated block identifier \\
\hline
{[raw block]} & Link to download the raw block JSON \\
\hline
{[proof]} & Link to the ZK proof artifact \\
\hline
Regime & Dominant behavioral pattern combination \\
\hline
Anomaly: 0.37 & Score from 0--1 (higher = more unusual) \\
\hline
Density & Peak hotzone density (KDE value) \\
\hline
PQ & Average PQ reconstruction error \\
\hline
\end{tabular}
\end{center}

\subsubsection{Anomaly Labels}

\begin{center}
\begin{tabular}{lll}
\toprule
\textbf{Score Range} & \textbf{Label} & \textbf{Meaning} \\
\midrule
$< 0.45$ & Typical & Normal activity \\
$0.45 - 0.75$ & Unusual & Notable patterns \\
$> 0.75$ & Rare & Potential anomaly \\
\bottomrule
\end{tabular}
\end{center}

\subsection{Predicted Next Block Card}

Shows predictions for the next block:
\begin{itemize}
    \item \textbf{Chain} --- Which blockchain (SOL, ETH, AVAX)
    \item \textbf{Confidence} --- How sure we are (e.g., 24\%)
    \item \textbf{Predicted Tags} --- Expected behaviors
    \item \textbf{ETA} --- Estimated time until next block
    \item \textbf{Reason} --- Why we made this prediction
\end{itemize}

\textbf{How predictions work:}
\begin{enumerate}
    \item Keep a rolling history of recent mempool snapshots
    \item Compute trends: gas slope, transaction delta, tag frequency changes
    \item If trends are stable $\rightarrow$ MIXED\_ACTIVITY with low confidence
    \item If gas spiking + DEX activity $\rightarrow$ HIGH\_FEE + DEX\_ACTIVITY
\end{enumerate}

\subsection{Mempool Ticker (Live Feed)}

Shows real-time pending transactions:
\begin{itemize}
    \item \textbf{Timestamp} --- When the snapshot was taken
    \item \textbf{Chain} --- Which blockchain
    \item \textbf{Pressure} --- High / Normal / Low
    \item \textbf{Stats} --- Pending tx count, gas price, total value
    \item \textbf{Highlights} --- Notable patterns detected
\end{itemize}

\textbf{Important:} This shows the \textbf{mempool} (pending transactions), NOT confirmed blocks.

\subsection{Block Detail Page}

When you click on a block:
\begin{itemize}
    \item \textbf{Hotzones} --- Up to 16 density clusters, each with tags
    \item \textbf{Hypergraph} --- 3D graph of hotzone relationships
    \item \textbf{Metrics} --- Transaction count, PQ residuals, commitment hash
\end{itemize}

\subsection{Atlas Page}

Shows all hotzones from recent blocks projected into 2D or 3D:
\begin{itemize}
    \item \textbf{Position} --- Coordinates from dimensionality reduction
    \item \textbf{Size} --- Proportional to density
    \item \textbf{Color} --- Mapped from dominant semantic tag
    \item \textbf{Edges} --- Hyperedges connecting related hotzones
\end{itemize}

\section{The Pipeline}

\subsection{Step 1: Block Ingestion}

\begin{lstlisting}
RPC Node --> watchIngest.js --> Raw Block JSON
\end{lstlisting}

The ingestion worker polls blockchain RPC endpoints:
\begin{itemize}
    \item \textbf{Ethereum} --- via ETH\_RPC\_URLS
    \item \textbf{Solana} --- via Solana RPC
    \item \textbf{Avalanche} --- via AVAX RPC
\end{itemize}

Downloads: block header, transactions, execution traces, state changes.

\subsection{Step 2: Vectorization}

\begin{lstlisting}
Raw Block --> vectorizeBlock() --> Vectors
\end{lstlisting}

Each transaction becomes a \textbf{16-dimensional vector}:
\begin{lstlisting}
Transaction {amount, fee, gas, ...} 
    --> Vector [0.72, 0.15, 0.33, 0.5, ...]
\end{lstlisting}

Encoded features: normalized amounts, fees, gas, contract types, assets, addresses, timestamps.

\subsection{Step 3: Folding}

\begin{lstlisting}
[1000+ vectors] --> foldVectors() --> [6 vectors]
\end{lstlisting}

Compress all vectors into 6 summary vectors:
\begin{enumerate}
    \item \textbf{Mean vector} --- Average of all transactions
    \item \textbf{Variance vector} --- How spread out values are
    \item \textbf{Components 1--4} --- Principal directions of variation
\end{enumerate}

Reduces 1000 transactions from $\sim$16,000 numbers to just 96.

\subsection{Step 4: Product Quantization (PQ)}

\begin{lstlisting}
[6 vectors] --> pqEncode() --> [6 codes] + residuals
\end{lstlisting}

Each vector compressed using a codebook:
\begin{lstlisting}
Vector [0.72, 0.15, 0.33, 0.5] 
    --> Split into 4 subvectors
    --> Find nearest centroid for each
    --> Code: [142, 87, 203, 56]
\end{lstlisting}

The \textbf{residual} measures compression error.

\subsection{Step 5: Hotzone Detection (KDE)}

\begin{lstlisting}
PQ codes --> pqDecode() --> Vectors --> KDE --> Hotzones
\end{lstlisting}

Run Kernel Density Estimation:
\begin{lstlisting}
For each vector:
    density = sum of Gaussian kernels
    
If density > threshold:
    This is a HOTZONE
\end{lstlisting}

Hotzones get semantic tags based on vector component values.

\subsection{Step 6: Hypergraph Construction}

\begin{lstlisting}
Hotzones --> buildHypergraph() --> Graph
\end{lstlisting}

Connect hotzones that are close in vector space and both high density.

\subsection{Step 7: Anomaly Scoring}

\begin{lstlisting}
(density, residuals, tags) --> computeAnomalyScore() --> 0.37
\end{lstlisting}

\begin{center}
\begin{tabular}{lll}
\toprule
\textbf{Component} & \textbf{Weight} & \textbf{Measures} \\
\midrule
Density & 50\% & Activity concentration \\
Residuals & 35\% & PQ compression quality \\
Tags & 15\% & Suspicious tag presence \\
\bottomrule
\end{tabular}
\end{center}

\subsection{Step 8: Cryptographic Commitment}

\begin{lstlisting}
Fingerprint --> BLAKE3 hash --> Commitment
\end{lstlisting}

Create tamper-proof commitments for folded vectors, PQ codes, and codebook.

\subsection{Step 9: ZK Proof Generation}

\begin{lstlisting}
Witness + Public Inputs --> Halo2 Prover --> Proof
\end{lstlisting}

Generate zero-knowledge proof that:
\begin{enumerate}
    \item Vectorization was correct
    \item Folding math is right
    \item PQ encoding respects error bound
    \item Commitments match
\end{enumerate}

\textbf{Current status:} Running mock proofs. Real proofs need compiled Halo2 binaries.

\subsection{Step 10: Storage and Display}

\begin{lstlisting}
Summary + Proof --> SQLite + JSON --> Dashboard
\end{lstlisting}

Saved to SQLite (for queries) and JSON files (raw artifacts).

\section{Services Architecture}

\subsection{Local Development}

\begin{center}
\begin{tabular}{lll}
\toprule
\textbf{Service} & \textbf{Port} & \textbf{Purpose} \\
\midrule
Next.js Dashboard & 3000 & Web UI \\
ngrok tunnel & --- & Public URL \\
Ingest Worker & --- & Fetches blocks \\
Mempool Worker & --- & Watches pending txs \\
\bottomrule
\end{tabular}
\end{center}

\subsection{Cloud Deployment}

\begin{center}
\begin{tabular}{lll}
\toprule
\textbf{Service} & \textbf{Platform} & \textbf{Purpose} \\
\midrule
API & Render & Serves data \\
Ingest Worker & Render & Block ingestion \\
Mempool Worker & Render & Mempool watching \\
Dashboard & Vercel & Next.js frontend \\
\bottomrule
\end{tabular}
\end{center}

\section{Key Files}

\begin{center}
\begin{tabular}{lp{7cm}}
\toprule
\textbf{File} & \textbf{Purpose} \\
\midrule
\texttt{scripts/ingestBlocks.ts} & Main ingestion pipeline \\
\texttt{scripts/mempoolWatch.ts} & Mempool polling + predictions \\
\texttt{folding/vectorize.ts} & Transaction to Vector \\
\texttt{folding/fold.ts} & Vectors to Folded Block \\
\texttt{folding/pq.ts} & Product Quantization \\
\texttt{analytics/hotzones.ts} & KDE + hotzone detection \\
\texttt{analytics/hypergraph.ts} & Graph construction \\
\texttt{zk/halo2Backend.ts} & ZK proof interface \\
\texttt{dashboard/app/page.tsx} & Main dashboard page \\
\bottomrule
\end{tabular}
\end{center}

\section{Quick Explanation Script}

\begin{shaded}
``YYSFold takes raw blockchain blocks and compresses them into behavioral fingerprints. Each block with thousands of transactions gets reduced to just 6 summary vectors using statistical folding and product quantization.

We then run kernel density estimation to find `hotzones'---clusters of similar transaction patterns---and tag them semantically (DEX activity, bridge transfers, potential AML alerts, etc.).

The anomaly score tells us if a block looks normal or unusual compared to baseline. The mempool feed shows pending transactions in real-time, and we use trend analysis to predict what the next block might look like.

Everything is cryptographically committed with BLAKE3 hashes, and we can optionally generate zero-knowledge proofs that the analysis was done correctly---so you can verify the fingerprint without re-running all the math yourself.''
\end{shaded}

\section{Glossary}

\begin{center}
\begin{tabular}{lp{8cm}}
\toprule
\textbf{Term} & \textbf{Definition} \\
\midrule
Fingerprint & Compact representation of a block's behavior \\
Folding & Compressing many vectors into few summary vectors \\
PQ & Product Quantization---vector compression using codebook \\
KDE & Kernel Density Estimation---finding dense regions \\
Hotzone & A cluster of similar transactions \\
Hypergraph & Graph where edges can connect 2+ nodes \\
Anomaly Score & 0--1 measure of how unusual a block is \\
Mempool & Pool of pending (unconfirmed) transactions \\
ZK Proof & Cryptographic proof of correct computation \\
Commitment & Hash that binds data without revealing it \\
SSE & Server-Sent Events---real-time data streaming \\
RPC & Remote Procedure Call---API to blockchain nodes \\
\bottomrule
\end{tabular}
\end{center}

\end{document}





